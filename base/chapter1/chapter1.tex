\setcounter{section}{-1}

\section{Preface}

In Chapter 1, we apply UMAP to population genetic data for the first time. Until this time, dimensionality reduction in population genetics was largely limited to PCA, with the occasional foray into methods like t-SNE. We provide an in-depth analysis and comparison of PCA, t-SNE, and UMAP on genotype data from three biobanks: the 1KGP, the HRS, and the UKB.

We explore a variety of visualization methods and illustrate the relative strengths of UMAP as well as its limitations compared to other methods. We use UMAP to reduce our data to $2$ dimensions and uncover fine-scale population structure in each of our data sets and colour it with sociodemographic data, geographic coordinates, phenotype distributions, admixture estimates, and other variables to reveal intricate patterns. We use UMAP to reduce our data to $3$ dimensions and translate this from $(x,y,z)$ coordinates to $(R,G,B)$ values to show how to use topological data analysis to reveal spatial gradients in population structure.

This manuscript became the basis of several UMAP analyses by other researchers in a wide variety of contexts. It is now standard for new biobanks to publish a UMAP plot of their population structure. This manuscript was released as a preprint on \textit{BioRxiv} in 2018 and published in \textit{PLoS Genetics} in 2019.

\section{Abstract}

Human populations feature both discrete and continuous patterns of variation. Current analysis approaches struggle to jointly identify these patterns because of modelling assumptions, mathematical constraints, or numerical challenges. Here we apply uniform manifold approximation and projection (UMAP), a non-linear dimension reduction tool, to three well-studied genotype datasets and discover overlooked subpopulations within the American Hispanic population, fine-scale relationships between geography, genotypes, and phenotypes in the UK population, and cryptic structure in the Thousand Genomes Project data. This approach is well-suited to the influx of large and diverse data and opens new lines of inquiry in population-scale datasets.

\section{Author summary}

The demographic history of human populations features varying geographic and social barriers to mating. Over time, these barriers have led to varying levels of genetic relatedness among individuals.  This population structure is informative about human history, and can have a significant impact on studies of medical genetics. Because population structure depends on myriad demographic, ecological, and social forces, a priori visualization is useful to identify subtle patterns of population structure. We use a dimension reduction method---UMAP---to visualize population structure in three genomic datasets and find previously unobserved patterns, revealing fine-scale population structure and illustrating differences between groups in traits such as white blood cell count, height, and FEV1, a measure of lung function. Using UMAP is computationally efficient and can identify fine-scale population structure in large population datasets. We find it particularly useful to reveal phenotypic variation among genetically related populations, and recommend it is a complement  to principal component analysis in primary data visualization. 

\section{Introduction}
Questions in medicine, anthropology, and related fields hinge on interpreting the deluge of genomic data provided by modern high-throughput sequencing technologies. Because genomic datasets are high-dimensional, their interpretation requires statistical methods that can comprehensively condense information in a manner that is understandable to researchers and minimizes the amount of data that is sacrificed. Both model-based and model-agnostic approaches to summarize data have played important roles in shaping our understanding of the evolution of our species (e.g., \citep{lawson2012inference, novembre2016recent, spence2018inference, eigen2006, Hellenthal747}).

Here we will focus on nonparametric approaches to visualize relatedness patterns among individuals within populations. If we consider unphased single nucleotide polymorphism (SNP) data, an individual genome can be represented as a sequence of integers corresponding to the number of copies of the alleles carried by the individual at each of the $L$ SNPs for which genotypes are available, with $L$ ranging from hundreds of thousands to hundreds of millions. Since each individual is represented as an $L$-dimensional vector, dimension reduction methods are needed to visualize the data.

Principal component analysis (PCA) is often the first dimensional reduction tool used for genomic data. It identifies and ranks directions in genotype space that explain most-to-least variance among individuals. Positions of individuals along directions of highest variance can then be used to summarize individual genotypes. PCA coordinates have natural genealogical interpretations in terms of expected times to a most recent common ancestor (TMRCA) \citep{mcvean2009genealogical}, and are used empirically to reveal admixture \citep{brisbin2012pcadmix}, continuous isolation-by-distance \citep{novembre2008europe, nelson2008population}, as well as technical artefacts. PCA coordinates are particularly well-suited to correct for population structure in GWAS\citep{eigen2006}.

The amount of information encoded in the highest-variance PCs increases slowly with sample size, so researchers typically examine multiple two-dimensional projections to lower-variance PCs to explore data. In this process, finer features of the data may be hidden by the projections or hard to interpret. To display finer features of the data in a two dimensional figure, we can use non-linear transformations that emphasize the local structure of the data. A popular method for such visualization is t-distributed stochastic neighbour embedding (t-SNE)\citep{maaten2008visualizing}. t-SNE has been used before to visualize SNPs\citep{platzer2013visualization}. Using data from the 1000 Genomes Project (1KGP)\citep{10002015global}, it groups individuals corresponding roughly to their continent of origin, with smaller ethnic sub-groups visible within the larger continental clusters\citep{li2017tsne}. However, t-SNE struggles with very large datasets, when a large number of locally optimal configurations make convergence to a globally satisfying solution difficult.

Uniform Manifold Approximation and Projection (UMAP) is a dimension reduction technique designed to model and preserve the high-dimensional topology of data points in the low-dimensional space\citep{2018arXivUMAP}. With genotype data, UMAP creates a neighbourhood around each individual's genetic coordinates and identifies a pre-selected number of neighbours to build high-dimensional manifolds. The end result is a patchwork of low-dimensional representations of neighbourhoods that groups genetically similar individuals together on a local scale while better preserving long-range topological connections to more distantly related individuals. The method has been successfully applied to single-cell RNA sequencing datasets \citep{umap2018singlecell}.

Non-linear dimension reduction methods tend to be computationally intensive. A common practice to reduce this burden is to first apply PCA to data, and perform dimensional reduction on data projected to leading principal components (PCs). In addition to being computationally advantageous, this discards noise that can confound non-linear approaches: population structure arising from $n$ isolated randomly-mating demes can be described by the leading $n-1$ PCs, with the following PCs describing stochastic variation in relatedness~\citep{eigen2006}. Selecting the leading PCs therefore has potential to extract meaningful population structure while filtering out stochastic noise. We explore different strategies to pre-process the data and investigate discrete and continuous population structure patterns present in large datasets of human genotypes: the 1KGP, the Health and Retirement Study (HRS)\citep{juster1995overview}, and the UK BioBank (UKBB)\citep{sudlow2015uk}, and compare UMAP's performance to t-SNE.  

\section{Results}
\subsection{Fine-scale visualization of the 1KGP dataset}



