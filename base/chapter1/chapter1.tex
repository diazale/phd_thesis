\setcounter{section}{-1}

\section{Preface}

In Chapter 1, we apply UMAP to population genetic data for the first time. Until this time, dimensionality reduction in population genetics was largely limited to PCA, with the occasional foray into methods like t-SNE. We provide an in-depth analysis and comparison of PCA, t-SNE, and UMAP on genotype data from three biobanks: the 1KGP, the HRS, and the UKB.

We explore a variety of visualization methods and illustrate the relative strengths of UMAP as well as its limitations compared to other methods. We use UMAP to reduce our data to $2$ dimensions and uncover fine-scale population structure in each of our data sets and colour it with sociodemographic data, geographic coordinates, phenotype distributions, admixture estimates, and other variables to reveal intricate patterns. We use UMAP to reduce our data to $3$ dimensions and translate this from $(x,y,z)$ coordinates to $(R,G,B)$ values to show how to use topological data analysis to reveal spatial gradients in population structure.

This manuscript became the basis of several UMAP analyses by other researchers in a wide variety of contexts. It is now standard for new biobanks to publish a UMAP plot of their population structure. This manuscript was released as a preprint on \texit{BioRxiv} in 2018 and published in \textit{PLoS Genetics} in 2019.

\section{Abstract}

Human populations feature both discrete and continuous patterns of variation. Current analysis approaches struggle to jointly identify these patterns because of modelling assumptions, mathematical constraints, or numerical challenges. Here we apply uniform manifold approximation and projection (UMAP), a non-linear dimension reduction tool, to three well-studied genotype datasets and discover overlooked subpopulations within the American Hispanic population, fine-scale relationships between geography, genotypes, and phenotypes in the UK population, and cryptic structure in the Thousand Genomes Project data. This approach is well-suited to the influx of large and diverse data and opens new lines of inquiry in population-scale datasets.

\section{Author summary}

The demographic history of human populations features varying geographic and social barriers to mating. Over time, these barriers have led to varying levels of genetic relatedness among individuals.  This \emph{population structure} is informative about human history, and can have a significant impact on studies of medical genetics. Because population structure depends on myriad demographic, ecological, and social forces, \emph{a priori} visualization is useful to identify subtle patterns of population structure. We use a dimension reduction method --- UMAP --- to visualize population structure in three genomic datasets and find previously unobserved patterns, revealing fine-scale population structure and illustrating differences between groups in traits such as white blood cell count, height, and FEV1, a measure used to detect airway obstruction. Using UMAP is computationally efficient and can identify fine-scale population structure in large population datasets. We find it particularly useful to reveal phenotypic variation among genetically related populations, and recommend it is a complement  to principal component analysis in primary data visualization. 

\section{Section 1}

Testing a citation\citep{novembre2008europe}