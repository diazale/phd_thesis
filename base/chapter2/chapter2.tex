\setcounter{section}{-1}

\section{Preface}

In the two years following the publication of a preprint of \hyperref[chap:chapter1]{Chapter~1}, UMAP gained widespread adoption in population genetics, being applied across many biobanks and to different types of genetic data, such as structural variants and ancient DNA. It had also been applied to animal data to study introgression, conservation genetics, and disease vectors. 

In this chapter, we review the applications of UMAP. We discuss the impacts of parametrizations on visualizations, the impacts of data filtering steps for LD and the human leukocyte antigen (HLA) region, and updates to the functionality of the Python implementation. We also discuss the use of UMAP in the context of exploratory data analysis.

This chapter was originally published in the \textit{Journal of Human Genetics} in 2020.

\clearpage

\section{Abstract}

Uniform manifold approximation and projection (UMAP) has been rapidly adopted by the population genetics community to study population structure. It has become common in visualizing the ancestral composition of human genetic datasets, as well as searching for unique clusters of data, and for identifying geographic patterns. Here we give an overview of applications of UMAP in population genetics, provide recommendations for best practices, and offer insights on optimal uses for the technique.

Chapter two. Chaaaapter two.\citep{wei_graph_2019}