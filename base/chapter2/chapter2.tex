\setcounter{section}{-1}

\section{Preface}

In the two years following the publication of a preprint of \hyperref[chap:chapter1]{Chapter~1}, UMAP gained widespread adoption in population genetics, being applied across many biobanks and to different types of genetic data, such as structural variants and ancient DNA. It had also been applied to animal data to study introgression, conservation genetics, and disease vectors. 

In this chapter, we review the applications of UMAP. We discuss the impacts of parametrizations on visualizations, the impacts of data filtering steps for LD and the human leukocyte antigen (HLA) region, and updates to the functionality of the Python implementation. We also discuss the use of UMAP in the context of exploratory data analysis.

This chapter was originally published in the \textit{Journal of Human Genetics} in 2020.

\clearpage

\section{Abstract}

Uniform manifold approximation and projection (UMAP) has been rapidly adopted by the population genetics community to study population structure. It has become common in visualizing the ancestral composition of human genetic datasets, as well as searching for unique clusters of data, and for identifying geographic patterns. Here we give an overview of applications of UMAP in population genetics, provide recommendations for best practices, and offer insights on optimal uses for the technique.

\section{Introduction}
One of the primary challenges of genomic data analysis is high dimensionality. The human genome has over three billion base pairs, and many biobanks contain hundreds of thousands of individuals and above. Relationships among individuals are relevant for historical studies as well as for studies that seek to identify genetic roots of diseases. These relationships can be influenced by demography, sampling strategies, and technical variation. A first step in many genomic analyses is dimensionality reduction to visualize the data to identify relevant relatedness patterns. 

One of the most common methods of dimensionality reduction is principal component analysis (PCA). PCA identifies directions, in the high-dimensional space, along which data is most variable. The projection of genomic data along these directions provides a low-dimensional representation that captures as much variance as possible. Because PCA projection is a linear operation, it has a relatively straightforward interpretation in terms of demographic events (i.e, distances between populations can be interpreted in terms of times to the most recent common ancestors) \citep{mcvean2009genealogical}.  It is also well-suited to the correction of population structure in genome-wide association studies (GWAS)\citep{patterson2006population}, and is therefore widely used.

Dimensionality reduction requires tradeoffs. Because PCA projection identifies directions of maximal variance in the data and ignores variation along other directions, it tends to obscure finer scale patterns of population structure. Many nonlinear neighbour graph-based dimension reduction algorithms, such as t-SNE\citep{maaten_visualizing_2008}, have been developed over the years to overcome this limitation. Here we focus on uniform manifold approximation and projection (UMAP)\citep{mcinnes_umap_2020}, a method developed in 2018 that has seen widespread use across fields (e.g. single-cell genomics\citep{becht2019dimensionality}). 

Rather than trying to preserve large-scale structure, UMAP seeks to preserve local neighbourhoods in a dataset. For each individual in a genetic dataset, UMAP identifies a pre-set number of nearest neighbours and represents distances to these neighbours as a weighted graph where the nearest neighbours are weighted more heavily. The goal is then to find a low-dimensional representation of the data that preserves these neighbourhoods as much as possible. By focusing on preserving neighborhood topology rather than absolute distances, UMAP allows for data-dense regions to be ``stretched out'' in the representation. This can have the benefit of reducing overcrowding of the low-dimensional representation, but comes at the cost of a more challenging interpretation of distances.  This is an important distinction relative to algorithms such as PHATE \citep{moon2019visualizing} that allow nonlinear transformations of the data while seeking to preserve meaningful distances. 

A consequence of the focus on topology is that the meaning of distances in the reduced space is difficult to interpret. Even though most nonlinear dimension reduction methods allow for some stretching of distances to improve visualization of local structure, UMAP can be thought of as particularly permissive, as it does not penalize uniform stretching. Because of this, UMAP representations can also contain arbitrarily small distances between points. Though such small distances might be a faithful representation of the original data topology, they are not ideal for visualization. UMAP allows for specification of a minimum distance between nearest neighbours in low-dimensional space: higher values are useful for visualization, but values near or equal to zero can be used for downstream analyses, such as clustering.

In the context of genetic data, UMAP finds the nearest genetic neighbours for each individual and creates low-dimensional representations that group more closely-related individuals together, and partially preserves longer-range relatedness through intermediary individuals. When used in visualizations, UMAP embeddings uncover many subtle features of data, such as distinct demographic histories and covariation between genetics, geography, and phenotypes\citep{diaz-papkovich_umap_2019}. Figure~\ref{fig:PCA_and_UMAP} compares visualizations of PCA to UMAP using genotype data from the Thousand Genomes Project (1000GP)\citep{global_2015}. PCA flattens the third dimension, obscuring the distinction between South Asian and Central/South American population clusters, whereas UMAP places them in more clearly visible clusters. UMAP has become widely used to study population structure in humans and other species, in conjunction with existing methods. Here we will describe the current state of the use of UMAP in population genetics.
