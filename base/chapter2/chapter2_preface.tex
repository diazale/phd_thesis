\setcounter{section}{-1}

\section{Preface}

In Chapter 2, we apply UMAP to population genetic data for the first time. Until this time, dimensionality reduction in population genetics was largely limited to PCA, with the occasional foray into methods like $t$-SNE. We provide an in-depth analysis and comparison of PCA, t-SNE, and UMAP on genotype data from three biobanks: the 1KGP, the HRS, and the UKB.

We explore a variety of visualization methods and illustrate the relative strengths of UMAP as well as its limitations compared to other methods. We use UMAP to reduce our data to $2$ dimensions and uncover fine-scale population structure in each of our data sets and colour it with sociodemographic data, geographic coordinates, phenotype distributions, admixture estimates, and other variables to reveal intricate patterns. We use UMAP to reduce our data to $3$ dimensions and translate this from $(x,y,z)$ coordinates to $(R,G,B)$ values to show how to use topological data analysis to reveal spatial gradients in population structure.

This manuscript became the basis of several UMAP analyses by other researchers in a wide variety of contexts. It is now standard for new biobanks to publish a UMAP plot of their population structure. This manuscript was released as a preprint on \textit{BioRxiv} in 2018 and published in \textit{PLoS Genetics} in 2019.

\clearpage