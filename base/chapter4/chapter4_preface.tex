\setcounter{section}{-1}

\section{Preface}

One common use of UMAP in population genetics is to identify clusters and to treat them as populations for downstream analysis. However, there is no effective way to algorithmically extract clusters from UMAP plots. Though it generates clusters visually, it is a dimensionality reduction algorithm and not a clustering algorithm. Centroid- or archetype-based approaches fail to capture many individuals, rely on arbitrary definitions of population groups, or require a pre-specified number of populations. Researchers often resorted to hand-delineating clusters, which is limited to $2$D projections and not scalable in the presence of many populations.

We apply HDBSCAN($\hat{\epsilon}$), a hierarchical density-based clustering algorithm to UMAP data. This approach can use UMAP embeddings of arbitrary dimensions---importantly allowing us to work in $3$ or more dimensions. Running on the order of seconds for massive biobanks, it creates topological clusters that reflect the demgraphic histories of populations. We apply the algorithm to three biobanks (the 1KGP, UKB, and CaG cohorts) and demonstrate its effectiveness at capturing population structure, usefulness in analysis of biobank data, potential downstream applications (e.g. for PGS transferability), and its use as a quality control tool.

This manuscript was released as a preprint on \textit{bioRxiv} in 2023.

\clearpage