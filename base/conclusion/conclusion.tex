This thesis has introduced a new methodology for high-dimensional population genetic data, applying UMAP and HDBSCAN($\hat{\epsilon}$) to several biobanks. In \hyperref[chap:chapter2]{Chapter~2} we applied UMAP to population genetic data for the first time, using it for visualization and revealing fine-scale population structure on data sets of up to half a million individuals. We illustrated its potential for exploratory analyses with examples of many applications; it is now a standard method in the field. In \hyperref[chap:chapter3]{Chapter~3}, we review the uses of UMAP in population genetics, discuss the impacts of data filtering and parametrization, and provide guidance on best practices. Finally, in \hyperref[chap:chapter4]{Chapter~4}, we provide a method to reliably extract clusters in UMAP data. We use these methods to stratify biobank data, characterize population structure, carry out quality control, study the impacts of PCA correction in GWAS and PGS and the transferability of PGS between different populations. In each chapter, we expound upon the relationships between UMAP data, clusters, and auxiliary data such as population labels, distributions of phenotypes, the geographic distribution of genetic variation, and demographic history. In each case, we have also made our code and data publicly available. 

In conclusion, this thesis has established a basis for deep exploration of biobank data using topological analysis. It is a methodology that is fast, tractable, creates rich visualizations that provoke investigation, and has potential for many follow-up studies.