Pre-amble to the conclusion goes here.

% note: convert these to chapters
\section{Discussion}

Inferential statistics requires a definition of ``population''\citep{statcan2003}. To establish the scope of a study, we define the ``target population'' (the group to which research applies) and the ``survey population'' (the group which is covered by the study and, ideally, is very similar to the target population). The survey population is ultimately chosen from the survey frame, which provides the means to sample the individuals from the population. [talk about popgen, biobanks, assumptions, etc]

PRS more like LOL\citep{kaplan_polygenic_2022}

Structure may be subtle and arise from, e.g., recent population structure or admixture\citep{gopalan_human_2022}

Bigger data is not better data. We cannot just collect our way out of this --- we need to understand what it is that we are doing and what we are working with.

test\citep{rose_sick_2001}

Though clustering is often viewed as a statistical or computational method, it has topological underpinnings as well.

genetics of participants may not match non-participants
\citep{benonisdottir_studying_2023}

``Is it possible to derive low dimensional embedding methods that explicitly preserve topological features of the data? This is an interesting open question.''\citep{wasserman_topological_2018}

\section{Future directions}

\section{Final Conclusion}
My Conclusions.