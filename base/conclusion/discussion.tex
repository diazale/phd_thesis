We have prese

We have presented a method of nonlinear dimensionality reduction that is compatible with scale and composition of modern biobanks. 

\section{The value of visualization}

Presenting this research provokes an interesting duality: people will ask if there is any value to the visualizations, while simultaneously implementing them. There is an inherent value to being able to see your data and understand its structure, even if the meaning does not jump out immediately. Echoing a point from Chapter 2, dimensionality reduction is to data what the microscope is to biology. We could certainly do biological research purely through statistical inference and clever theory, but it would not be nearly as fast or fruitful or creative.

- the philosophy of unsupervised learning and clustering in general
- we're finding something, it's probably not nothing

Creative uses like colouring maps

\section{Defining populations}

Inferential statistics requires a definition of ``population''\citep{statcan2003}. To establish the scope of a study, we define the ``target population'' (the group to which research applies) and the ``survey population'' (the group which is covered by the study and, ideally, is very similar to the target population). The survey population is ultimately chosen from the survey frame, which provides the means to sample the individuals from the population. [talk about popgen, biobanks, assumptions, etc]


Structure may be subtle and arise from, e.g., recent population structure or admixture\citep{gopalan_human_2022}

Bigger data is not better data. We cannot just collect our way out of this --- we need to understand what it is that we are doing and what we are working with.



Though clustering is often viewed as a statistical or computational method, it has topological underpinnings as well.

genetics of participants may not match non-participants
\citep{benonisdottir_studying_2023}

``Is it possible to derive low dimensional embedding methods that explicitly preserve topological features of the data? This is an interesting open question.''\citep{wasserman_topological_2018}

The UKB's nested structure, and how definitions shape our perceptions.

\section{Equitability in the genomic revolution}

PRS more like LOL\citep{kaplan_polygenic_2022}

\section{The genetics-environment interplay}

Sick people and sick populations\citep{rose_sick_2001}

Biases in biobanks?

\section{History of the field}

The meanings of structure, biological essentialism, and the shadow of eugenics.

What we show undermines these narratives---there are no true clusers, no true populations, no perfect models or representations, but there are \textit{useful} ones.

Bad actors are not motivated by methodology---the correlation coefficient, the density plot, and the scatter plot remain common tools in spurious claims. Still we must take care in the way we present our research, understanding that it can easily be stripped of context.

Algorithmic colonialism, helicopter/imperialistic research, 

\section{Limitations}

