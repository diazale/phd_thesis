We have presented a method of nonlinear dimensionality reduction that is compatible with the scale and composition of modern biobanks. With UMAP and HDBSCAN($\hat{\epsilon}$) we have uncovered a wide variety of patterns of fine-scale population structure in every biobank studied. Since the publication of \hyperref[chap:chapter2]{Chapter~2} , UMAP has become a standard method for visualization and exploratory data analysis in population genetics. Given the demand for cluster extraction noted in \hyperref[chap:chapter4]{Chapter~4}, it is possible that HDBSCAN($\hat{\epsilon}$)---or some other density clustering method---could see similar adoption.

The timing of UMAP coincides fortuitously with the growth of biobanks and the genomic revolution. Writing in his review of TDA in 2018, Wasserman asked, ``is it possible to derive low dimensional embedding methods that explicitly preserve topological features of the data? This is an interesting open question.''\citep{wasserman_topological_2018} This review was published in the same year that UMAP, an explicitly topological method, was released. Topological analysis and density clustering seem especially well-suited to the task of understanding population structure in diverse biobanks.

\section{The value of visualization}

In \hyperref[chap:chapter3]{Chapter~3} we describe the relationship between dimensionality reduction and biological data as analogous to that of microscopes and biological samples. This analogy has been made before with respect to mathematics in general\citep{cohen_mathematics_2004}. Though analyses, inferences, and models can shed light on the biological world, there is an inherent value to being able to literally see your data and understand its structure. Among the most cited papers in population genetics is ``Genes mirror geography within Europe''\citep{novembre2008europe}---although isolation-by-distance had been characterized almost a century earlier, the figure of the top principal components superimposed over Europe was an elegant illustration of the geographical distribution of human genetic variation.

With UMAP, we are able to map complex genetic structure down to $2$ or $3$ dimensions for visualization, examine the fine-scale structure that a method like PCA compresses, while preserving interesting signal in the data. Using auxiliary data---geographic data, population labels, phenotypic information, environmental variables, etc.---we can visually scan for patterns and generate hypotheses. In \hyperref[chap:chapter2]{Chapter~2} we present a variety of visualization methods, including using colouring points by admixture levels to uncover gradients in ancestry and converting 3D UMAP coordinates to RGB colour levels for colouring maps. One common method used in single-cell studies is to colour UMAP plots by gene expression levels (e.g. \citep{jessa_stalled_2019} Figure~4h). A similar approach could be used in exploring allele distributions by clusters to see whether they are relatively more common in different populations---one such method, called the Allele Dispersion Score, was proposed by Correard et al. in 2022\citep{correard_allele_2022}. Though UMAP figures are now standard in genetic studies, most are limited to a simple 2D scatterplot coloured by some population label---in every biobank, there is certainly value in exploring deeper and using other variables.

There is also value in going beyond static $2$D figures. Software libraries such as plotly\citep{plotly} can generate interactive figures, are available for Python and R, and are straightforward to implement. Interactive exploration can assess how individual points fit within larger projections, rapidly identifying patterns or using them for diagnostics. Working in $3$D is much easier when the figures are interactive, and we have, e.g., explored the relationships between individuals and populations in the UKB. Finally, as noted in \hyperref[chap:chapter4]{Chapter~4}, clustering works well in dimensions abvoe $2$---while it can be tempting to delineate clusters from $2$D figures alone, the jump from $3$ to $2$ dimensions can lose information about how points are related to one another topologically. There is value in using information from higher dimensions to aid visualization in lower dimensions and, despite UMAP's now-widespread use, its ability to work in $3$ or more dimensions is, in the view of the author, underappreciated.

\section{The entangled biobank}

Biobanks are as complex as they are rich. Across visualizations in \hyperref[chap:chapter2]{Chapters~2} and \hyperref[chap:chapter4]{4}, we see complex patterns. Some appear quickly at the population level, as in Figure~\ref{fig:umap_height_female} where we see population-level differences in height; others require iterative smoothing, such as  Figure~\ref{fig:smoothed_phenos} where we see residual structure in phenotypes after removing the effects of the top $40$ PCs. These patterns also appear in behavioural measures, such as smoking in Figure~\ref{fig:supp_ukb_smoking}. Adding geographic data,migration histories,  and other variables to the mix, the difficulty of disentangling the genetic and environmental effects becomes more evident.





 while allowing the viewer to maintain some idea of the scale of the populations in mind. 

-bias in biobanks
-collider bias
-complexity
-some signal definitely exists across populations (selection for sickle cell, e.g., high altitude, deep diving, etc) but there is just so much variation in other variables that it's hard to keep track of

\clearpage

\section{Clusters and populations}

\begin{figure}[h]
\centering
    \includegraphics[width=0.75\linewidth]{main_figures/discussion/magritte.png}
\caption[The treachery of clustering]{\textbf{The treachery of clustering.} Clusters and even the idea of a population are concepts that help us understand reality, but are not reality itself.}
\label{fig:magritte}
\end{figure}

Unlike problems like classification, clustering does not have a well-defined ground truth, and even the most basic definition, such as ``similar points are grouped together and dissimilar points are grouped separately'', can be self-contradictory\citep{ben-david_clustering_2018}. In a genetic cluster from \hyperref[chap:chapter4]{Chapter~4}, individuals in may be closely related to their immediate neighbours, but not to those in a far end of a cluster; they are grouped by similarity in one sense, but not separated by dissimilarity. Different inputs, algorithms, and parametrizations will yield different results. Though various quality metrics exist, even untrained humans excel at assessing arbitrary $2$D clustering\citep{lewis_human_2012}---there is an element of ``I know it when I see it''. 

The question of the appropriateness of a given clustering can be framed as whether it is useful in its context: whether it identifies interesting patterns (without necessarily assigning interpretation); whether it simplifies data; investigation correlations with other variables, etc\citep{hennig_what_2015}. The choice of clustering method is often arbitrary, e.g. ``we use the standard method in the field''\citep{ben-david_clustering_2018}, and this has resulted in standards that exclude individuals from analyses\citep{ding_polygenic_2023}. We have demonstrated that density clustering successfully includes all or almost all individuals in several biobanks and that the clusters generated reflect logical groupings.
\citep{hennig_what_2015}

Though UMAP appears to preserve structure, at least visually, it is only one representation of very complex high-dimensional data points and the relationships between them. HDBSCAN($\hat{\epsilon}$) extracts clusters that correspond to our knowledge of population structure, but again they are only an abstraction and should not be reified. They are abstractions that, ideally, represent some underlying truth---they are maps of an unknown (unknowable?) territory; this map cannot be confused with the territory itself. 



``Empirical'' is not the same as ``objective''

\citep{lewis_human_2012}

This is a question of epistemology---

There is no universal definition of a ``cluster'', and the idea of a ``good'' clustering will be domain-specific\citep{hennig_what_2015}.

- discrete groupings are useful, like models --- and like models, they are prone to reification: the territory is not the map
- the philosophy of unsupervised learning and clustering in general
- we're finding something, it's probably not nothing


\clearpage

\section{Defining populations}

\begin{figure}[h]
\centering
\begin{subfigure}{0.45\linewidth}
    \includegraphics[width=\linewidth]{main_figures/discussion/populations_1.png}
    \caption{}
    \label{fig:statistical_populations1}
\end{subfigure}
\hfill
\begin{subfigure}{0.45\linewidth}
    \includegraphics[width=\linewidth]{main_figures/discussion/populations_2.png}
    \caption{}
    \label{fig:statistical_populations2}
\end{subfigure}
\caption[The relationship between target and study populations]{\textbf{The study population should match the target population.} In \textbf{(a)}, the study population is a subset of the target population, so inferences drawn from it will apply to the target population. In \textbf{(b)}, the study population is not fully a subset of the target population, so its inferences will be less reliable.}
\label{fig:statistical_populations}
\end{figure}

Though genetic variation exists on a multidimensional continuum, defining discrete populations can be useful, e.g. for modelling ancestry\citep{pritchard_inference_2000}. In inferential statistics, we also require a definition of ``population''\citep{statcan2003}. To establish the scope of a study, we define the ``target population'' (the group to which research applies) and the ``survey population'' (the group which is covered by the study and, ideally, is very similar to the target population). The survey population is ultimately chosen from the survey frame, which provides the means to sample the individuals from the population. Figure~\ref{fig:statistical_populations} provides a schematic of this concept, as well as a schematic of when the study population is not drawn from the target population.



Pgh1: Biobanks are the sources of biomedical studies, but how do they define their study populations and how do they define their target populations? Also point out that the studies done are not random samples from the target population but volunteer-based non-random samples. Then mention some of the associates found between markers and participationi n the UKB.

Pgh2: The way we define populations is pretty archaic and not great. Segue into hte next section (subsection?) where we discuss the histories of these ideas

Biobanks are the source of many biomedical studies, particularly GWAS and PGS. 

This principle applies to inferences and studies based on biobanks. 

In practice, biobank inference is often based on a form of non-probability sampling. 

-COB, ethnicity, biobank selection and ascertainment bias, etc

[talk about popgen, biobanks, assumptions, etc]


Structure may be subtle and arise from, e.g., recent population structure or admixture\citep{gopalan_human_2022}

Bigger data is not better data. We cannot just collect our way out of this --- we need to understand what it is that we are doing and what we are working with.



Though clustering is often viewed as a statistical or computational method, it has topological underpinnings as well.



\section{Implications for GWAS and PGS}

One of the promises of genomics is deciphering the genetic architecture of complex traits and illness. Unfortunately, the overwhelming majority of studies have been limited to European populations\citep{fatumo_roadmap_2022}\citep{kaplan_polygenic_2022}. Some oft-cited concerns include confounding from population structure and a lack of power from sample size of non-European individuals\citep{ben-eghan_dont_2020}. 

used k-means\citep{sun_analyses_2022}

genetics of participants may not match non-participants
\citep{benonisdottir_studying_2023}

The UKB's nested structure, and how definitions shape our perceptions.

How PC adjustment affects GWAS.

Don't just filter out ``other'' groups

Sick people and sick populations\citep{rose_sick_2001}

Biases in biobanks?

The meanings of structure, biological essentialism, and the shadow of eugenics.

What we show undermines these narratives---there are no true clusers, no true populations, no perfect models or representations, but there are \textit{useful} ones.

Algorithmic colonialism, helicopter/imperialistic research, 

\citep{gebru_race_2020}


\section{Limitations}

Dimensionality reduction necessarily sacrifices information. To help us navigate the physical world, we use two-dimensional Euclidean maps of a three-dimensional spherical planet. This transformation introduces artificial tearing and distortion into world maps---PCA, $t$-SNE, UMAP, and density clustering are no different. We may see patterns or clusters that are artefactual, arising from unidentified batch effects or the distribution of genetic variants particular to a biobank. Patterns may also form from noise. 

UMAP represents topology by approximating the local structure of the data. By patching together many local topological representations, it achieves a semblance of the global structure of data; however, the distances themselves do not have meaningful interpretations. PCA, which explicitly models global variation in data, has been interpreted in connection to TMRCA and F-statistics\citep{mcvean2009genealogical,peter_geometric_2022}. Several other non-linear dimensionality reduction methods aim to balance global and local structure (e.g. diffusion PHATE\citep{moon2019visualizing}, POPVAE\citep{battey_visualizing_2021}). UMAP captures topology, not geometry; it approximates the shapes of data, but now how the shapes are positioned relative to one another. Consider three nested hollow spheres in $3$D: each sphere forms one connected shape, but a $2$D UMAP would represent them as three disconnected clusters\citep{herrmann_enhancing_2022}. This correctly identifies the three connected components but not their relationships to each other.

\section{Future directions}

TDA has proven a useful addition to the population geneticist's toolbox. The work done in this thesis has been on genotypic data, and usually using the top principal components for computational reasons. Since the publication of the manuscripts, multi-threaded versions of UMAP and HDBSCAN($\hat{\epsilon}$) have been released. One avenue for future research is to work directly with genotype data that has not been processed with PCA. We explored this briefly in \hyperref[chap:chapter2]{Chapter~2} using 1KGP data; the improvement in compute time with multi-threading should make deeper investigation more feasible.

There are possible connections between genetic topology and IBD. As noted in \hyperref[chap:chapter4]{Chapter~4}, UMAP and HDBSCAN($\hat{\epsilon}$) extract population structure that appears similar in scale to IBD studies of diverse biobanks; the two methods may capture similar information but via different routes. Clusters extracted by our TDA approach likely contain substructure themselves, and a clustering approach that breaks clusters down further may reveal finer details.

Studying different subsets of populations or genetic data may be illuminating. We have noted studies in \hyperref[chap:chapter3]{Chapter~3} that use, e.g., structural variants rather than genotype data. Sex-bias in admixture has been noted in other studies (e.g. \citep{ongaro_evaluating_2021,korunes_sex-biased_2022,marcheco-teruel_cuba_2014}), whereas all of the analyses presented here have been on autosomal data and combining the sexes. Each of these approaches seems likely to uncover interesting patterns.

Within the biomedical realm, the confounding of GWAS and transferability of PGS are perennial areas of research. We presented a method of visualization in \hyperref[chap:chapter4]{Chapter~4} that illustrates how PCA adjustment affects populations across an entire biobank, with an additional analysis that studied the transferability of PGS. These approaches could also extend to studying the interplay between environment, genetics, and biomedical traits.