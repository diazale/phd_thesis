%\begin{figure}
%  \includegraphics[width=0.25\linewidth]{main_figures/test_figure.png}
%  \caption{A figure.}
%  \label{fig:test_fig}
%\end{figure}
%
%\newpage


\section{Overview}



%Refer to Figure~\ref{fig:test_fig}

\section{Population structure}

\subsection{The history of clustering}

\section{The shape of data}

\section{Data}
We use biobanks/cohorts.

\subsection{The 1000 Genomes Project}

The 1000 Genomes Project (1KGP) is a publicly available data set of genetic data sampled from many populations from around the world\citep{global_2015}. We used $3,450$ genotypes from the Affy 6.0 platform sampled from $26$  populations. The populations sizes are roughly similar, with between $104$ to $183$ in each group.

\subsection{CARTaGENE}

CARTaGENE (CaG) is a cohort of residents of Qu\'{e}bec with genotype data for $29,337$ participants, who were recruited using registration data from the R\'{e}gie de l’assurance maladie du Qu\'{e}bec (RAMQ), the provincial health authority\citep{awadalla_cohort_2013}. In addition to genetic data, it contains questionnaire health data and demographic information such as country of birth and ethnicity.

\subsection{Health and Retirement Study}

The Health and Retirement Study (HRS) is a cohort of retired American individuals\citep{juster_overview_1995}. We used genotype data from 12,454 individuals from the Health and Retirement Study (HRS), genotyped on the Illumina Human Omni 2.5M platform. The database contains basic demographic data such as age, US Census Bureau region of birth, and race.

\subsection{UK biobank}

The UK biobank (UKB) is a cohort of individuals living in the United Kingdom who were recruited by inviting those registered with the National Health Service (NHS)\citep{sudlow_uk_2015}. It contains the genotypes from $488,377$ participants as well as detailed health data, phenotypic measures, geographic coordinates, and sociodemographic information such as ethnic background.

\section{Methods}

\subsection{PCA}

\subsection{UMAP}

\subsection{HDBSCAN(\texorpdfstring{$\hat{\epsilon}$}{f})}
% the {f} here doesn't do anything, but we need a text character for compilation

\subsection{Genomic tools}