%\begin{figure}
%  \includegraphics[width=0.25\linewidth]{main_figures/test_figure.png}
%  \caption{A figure.}
%  \label{fig:test_fig}
%\end{figure}
%
%\newpage
	
\begin{quote} 
\begin{singlespace}
\textit{
\textbf{Fry:} Here you are in the year 3000 or so, yet you just sit around like it's the boring time I came from.\\
\textbf{Professor Farnsworth:} Boring?! Wasn't that the period when they cracked the human genome, and boy bands roamed the earth? \\
---Futurama, S03E11 (2001)}
\end{singlespace}
\end{quote}

In 2001, the first analyses of the human genome were published in sister papers in \textit{Nature} and \textit{Science}, ushering in a new era in biology. The Human Genome Project had been budgeted US\$$3$~billion in 1990; by 2020 the cost of sequencing a human genome had dropped to US\$$1,000$, and a relative paucity of data has given way to a deluge\citep{gibbs_human_2020}. Biobanks with data from tens or hundreds of thousands of individuals are becoming commonplace.

Every genome carries both the stories of its ancestors and the basic programming of its bearer's physiology. By identifying patterns across many genomes and their associated data, we can infer their histories and study distributions of biomedical traits. The complexities of human history and society, to say nothing of the complexities of biology itself, ensure that this is a non-trivial task.

With each genome spanning $3$~billion base pairs, any mathematical investigation is high-dimensional. This thesis explores advances in and applications of dimensionality reduction to population genetic data.

\section{Overview}

This thesis is organized into three chapters. In \hyperref[chap:chapter1]{Chapter~1}

%Refer to Figure~\ref{fig:test_fig}

\section{Population structure}

\subsection{The history of clustering}

\section{The shape of data}

\section{Data}
We use biobanks/cohorts.

\subsection{The 1000 Genomes Project}

The 1000 Genomes Project (1KGP) is a publicly available data set of genetic data sampled from many populations from around the world\citep{global_2015}. We used $3,450$ genotypes from the Affy 6.0 platform sampled from $26$  populations. The populations sizes are roughly similar, with between $104$ to $183$ in each group.

\subsection{CARTaGENE}

CARTaGENE (CaG) is a cohort of residents of Qu\'{e}bec with genotype data for $29,337$ participants, who were recruited using registration data from the R\'{e}gie de l’assurance maladie du Qu\'{e}bec (RAMQ), the provincial health authority\citep{awadalla_cohort_2013}. In addition to genetic data, it contains questionnaire health data and demographic information such as country of birth and ethnicity.

\subsection{Health and Retirement Study}

The Health and Retirement Study (HRS) is a cohort of retired American individuals\citep{juster_overview_1995}. We used genotype data from 12,454 individuals from the Health and Retirement Study (HRS), genotyped on the Illumina Human Omni 2.5M platform. The database contains basic demographic data such as age, US Census Bureau region of birth, and race.

\subsection{UK biobank}

The UK biobank (UKB) is a cohort of individuals living in the United Kingdom who were recruited by inviting those registered with the National Health Service (NHS)\citep{sudlow_uk_2015}. It contains the genotypes from $488,377$ participants as well as detailed health data, phenotypic measures, geographic coordinates, and sociodemographic information such as ethnic background.

\section{Methods}

\subsection{PCA}

\subsection{UMAP}

\subsection{HDBSCAN(\texorpdfstring{$\hat{\epsilon}$}{f})}
% the {f} here doesn't do anything, but we need a text character for compilation

\subsection{Genomic tools}