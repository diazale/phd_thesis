%\begin{figure}
%  \includegraphics[width=0.25\linewidth]{main_figures/test_figure.png}
%  \caption{A figure.}
%  \label{fig:test_fig}
%\end{figure}
%
%\newpage

In 2001, the first analyses of the draft human genome were published in sister papers in \textit{Nature} and \textit{Science}. The Human Genome Project had been budgeted US\$$3$~billion in 1990; by 2020 the cost of sequencing a human genome had dropped to US\$$1,000$, and a relative paucity of data had given way to abundance\citep{gibbs_human_2020}. Biobanks with data from tens or hundreds of thousands of individuals are becoming commonplace.

Every genome carries both the stories of its ancestors and the basic programming of its bearer's physiology. By identifying patterns across many genomes and their associated data, we can infer their histories and study distributions of biomedical traits. The complexities of human history and society, to say nothing of the complexities of biology itself, ensure that this is a non-trivial task.

With each genome spanning $3$~billion base pairs, any mathematical investigation is high-dimensional. This thesis explores applications of dimensionality reduction to population genetic data, focusing on uniform manifold approximation and projection (UMAP), a method of topological data analysis.

This thesis is organized into three chapters that have been published as stand-alone manuscripts. In \hyperref[chap:chapter1]{Chapter~1} we apply UMAP to human genetic data for the first time. We use genotype data from three biobanks, generating visualizations and observing patterns in relatedness, demographic histories, geographic distribution, phenotype distributions, and other phenomena. In \hyperref[chap:chapter2]{Chapter~2} we review the applications of UMAP in other human genetic datasets, such as different biobanks or other types of genetic data (e.g. structural variants). Finally, in \hyperref[chap:chapter3]{Chapter~3} we formalize a methodology to use UMAP in higher dimensions ($n \ge 3$) and extract clusters.

% high-level concepts?
% shape of data
% population structure
% complexity and interrelatedness

\section{Genetic diversity}

The human genome spans over $3$~billion nucleobase pairs organized across $23$ pairs of chromosomes---$22$ pairs of autosomes and one pair of sex chromosomes---with some DNA present in mitochondria (mtDNA). The human genome is diploid (i.e. paired) with one set of chromosomes coming from each parent via their gametes; these chromosomes are created through the process of meiotic recombination, in which the chromosomes of grandparents are aligned, cross over, and recombine. Along with mutation, recombination generates diversity. Approximately $99.9\%$ of DNA shared between humans is identical, with genetic variants (alleles) arising through mutations. Single nucleotide polymorphisms (SNPs) are relatively common variants, usually defined as having a frequency above $1\%$.

Variants that lie along the same chromosome and are not broken up through recombination are co-inherited and are linked. The block of allelic states along a DNA molecule is referred to as a haplotype, and when the same variants exist between two individuals, they are said to be identical by state (IBS). If the shared variant is inherited from a common ancestor without recombination, they are also said to be identical by descent (IBD); alleles that are IBS are typically IBD, with rare exceptions. Alleles that are physically closer are more likely to be inherited together, and those that appear together more often than expected at random are said to be in linkage disequilibrium (LD). Combining two haplotypes gives a diploid genotype, and assuming free recombination, the theoretical maximum number of possible unique haplotypes is $2^L$, where $L$ is the number of SNPs.

Recombination is not uniformly random. DNA that does not lie in the pseudoautosomal regions (PAR1 and PAR2) of the Y chromosome, as well as mtDNA, does not recombine\citep{jobling_human_2013}. Recombination rates also vary within chromosomes with certain regions known to be hotspots\citep{altemose_map_2017}. Germline mutations may result from copying errors during replication or from spontaneous errors from DNA’s instability or external factors like UV radiation. Whole genome sequencing pedigree-based studies estimate the overall mutation rate at about $10^{-8}$ per base pair per generation, though this rate may vary depending on the mechanistic source of the mutation\citep{segurel_determinants_2014}.

\subsection{Population structure}

% what does structure *look* like?

In aggregate, patterns will form in the distributions of alleles; these patterns are known as population structure. Though genetic diversity is random in principle, the distributions of variants are shaped by environments and events. Phenomena like migrations, population bottlenecks, and non-random mating contribute to the distribution of alleles.

Why is genetic variation not fully random? How do we measure it? What are the processes that we know of? What are some historical examples?

\subsection{Measures of population structure}

\subsubsection{Hardy-Weinberg Equilibrium}

\subsubsection{Fixation index}

\section{Exploratory data analysis}

Traditional statistical analyses of biological data follow the recommendations of R.A. Fisher in the 1930s\citep{holmes_modern_2019}. They have a linear structure: beginning with a biological question, an investigator forms a hypothesis and an associated null ($H_0$), designs an experiment, collects data, tests $H_0$ with a p-value, and formulates a conclusion. Beginning in the 1970s, statistician John W. Tukey proposed the alternative framework of exploratory data analysis (EDA)\citep{tukey_1977,hoaglin_john_2003}. This approach is iterative and instead begins with the data: we visualize it, understand it, and use it to inform what sort of analysis to use in an accompanying confirmatory data analysis. The two contrasting approaches are schematized in Figure~\ref{fig:paradigms}.

\begin{figure}[h!]
\centering
\begin{subfigure}{0.3\textwidth}
    \includegraphics[height=0.5\textheight]{main_figures/intro/fisher_paradigm.png}
    \caption{Fisher's paradigm.}
    \label{fig:fisher}
\end{subfigure}
\hfill
\begin{subfigure}{0.65\textwidth}
    \includegraphics[height=0.5\textheight]{main_figures/intro/tukey_paradigm.png}
    \caption{Tukey's paradigm.}
    \label{fig:tukey}
\end{subfigure}
\caption{\textbf{Contrasting Fisher's paradigm with Tukey's paradigm in biology.} Fisher's paradigm (left) takes a sequential approach to data analysis, beginning with a well-defined question and strong assumptions. Tukey's paradigm (right) is iterative, beginning with the data, emphasizing exploratory analysis through visualization, and complemented by confirmatory analyses that are robust and do not rely on complex assumptions\citep{holmes_modern_2019}.}
\label{fig:paradigms}
\end{figure}

Writing in 1980, Tukey emphasized that science neither begins with a tidy question nor ends with a tidy answer\citep{tukey_we_1980}. This is especially true in modern biology. Statistical questions from the 1930s typically had a few parameters $p$ with a manageable sample size $N$ (where $N > p$), and the people posing questions were involved in data collection. Today we sit at the opposite extreme; it is not unusual for data to have $p >> N$ with the two values differing by orders of magnitude. When studying a biobank, we may have several thousand individuals and several hundred thousand genetic markers. Generally, the people investigating data have not collected it. These factors make Tukey's paradigm much better suited to our analytical needs\citep{holmes_modern_2019}. 

\subsection{Dimensionality reduction}

In Figure~\ref{fig:tukey}, the iterative process includes presentation choices, graphics, and form recognition. This provokes a natural question: what approaches ought we use here? With genomic data comes the ``curse of dimensionality'': though we have many dimensions to our data, the signal is sparse and many methods are computationally intractable. This motivates dimensionality reduction---we wish to reduce our data to a relatively low number of dimensions, ideally preserving important characteristics of the data. Given a satisfactory representation of the data set, we can visualize it.

\subsubsection{Principal component analysis}
Principal components analysis (PCA) is a non-parametric linear transformation that projects data onto a series of orthogonal axes based on a linear combination of the original data. The axes are generated and ordered according to their eigenvalues, and the ratio of each axis' corresponding eigenvalue to the sum of all eigenvalues represents the variance explained by that axis. PCA fits an ellipsoid around the data in high dimensions and the axes of that ellipsoid are the principal components. By only selecting the largest axes---corresponding to the most explained variance --- we can reduce the dimensionality of our data while preserving significant explanatory value. We can also interpret our dimensionally reduced data in terms of how much of the overall variance it explains. Principal components are calculated through eigendecomposition of the covariance matrix; a derivation for genotype data is given in \hyperref[appendix:AppendixA]{Appendix~A}. A detailed examination of PCA in the context of population genetics can be found in \citep{mcvean_genealogical_2009}. 

PCA has seen wide application in population genetics. The top PCs often reflect isolation-by-distance and are used for visualization (e.g. within Europe\citep{novembre2008europe}). However, using them for visualization requires selecting which components to examine and is limited to $2$ or $3$ dimensions; if there is signal beyond the first few PCs, it may go unnoticed. We expand on this in \hyperref[chap:chapter1]{Chapter~1}.

They are also used to correct for population structure in genome-wide association studies (GWAS) by their inclusion as covariates in models\cite{price_principal_2006}. There are varying rules-of-thumb on how many PCs to include in a model, such as using the top $10$, looking for an ``elbow'' in the scree plot, or testing for eigenvalue significance in the Wishart distribution; however, these are merely conventions. We explore the impact of PC adjustment for phenotypes in biobanks in \hyperref[chap:chapter3]{Chapter~3}. 

\subsection{Topological data analysis}

% Note to self: Rewrite this in terms of popgen challenges
% can maybe get philosophical here
We are often interested in learning about our data in to understand its large-scale structure, e.g., identifying different cell types or related individuals. Though we have some definitions of distances, we are interested in notions of \textit{similarity} or \textit{nearness}. Topology provides the mathematical machinery for ideas rooted in qualitative geometry\citep{carlsson_topology_2009}. Topological data analysis (TDA) is a set of statistical methods that uses ideas of shape and connectivity to study data\citep{wasserman_topological_2018}. We will focus on applications of manifold learning, non-linear dimensionality reduction, and density clustering.

In TDA, we assume that we observe a sample $X_1, \dots, X_n \sim P$ with $P$ supported on some set $\supp(P) = \mathcal{X} \subseteq \mathbb{R}^d$. Suppose that $P$ is actually supported on some set $S$ with dimension $r$, where $r < d$ and $S$ is a smooth and compact manifold. Through manifold learning, we may estimate $S$; PCA can be considered a special case of manifold learning where data are assumed to lie on or near an affine subspace\citep{wasserman_topological_2018}. In population genetics, we observe samples from, e.g., the distributions of genotypes.

\subsubsection{t-distributed stochastic neighbour embedding}



\citep{maaten_visualizing_2008}

\subsection{Uniform manifold approximation and projection}

\subsection{Visualization}

% some principles of visualization?

% unsupervised learning?

\subsection{Clustering}

\citep{mcinnes_accelerated_2017}

\section{Data}

This research makes use of data from four biobanks. We focus on genotype data coded as the number of non-reference alleles. Given a set of $L$ SNPs for $N$ individuals, the genotype matrix $G$ is:

%$$
%G = \begin{bmatrix} 
%    g_{11} & g_{12} & \dots g_{1L} \\
%    \vdots & \ddots & \\
%    g_{N1} &        & g_{NL} 
%    \end{bmatrix}
%    
%\text{ where } g_{ij}\text{ is the number of non-reference alleles for individual } i \text{ at locus } j.
%$$

\subsection{The 1000 Genomes Project}

The 1000 Genomes Project (1KGP) is a publicly available data set of genetic data sampled from many populations from around the world\citep{global_2015}. We used $3,450$ genotypes from the Affy 6.0 platform sampled from $26$  populations. The populations sizes are roughly similar, with between $104$ to $183$ in each group.

\subsection{CARTaGENE}

CARTaGENE (CaG) is a cohort of residents of Qu\'{e}bec with genotype data for $29,337$ participants, who were recruited using registration data from the R\'{e}gie de l’assurance maladie du Qu\'{e}bec (RAMQ), the provincial health authority\citep{awadalla_cohort_2013}. In addition to genetic data, it contains questionnaire health data and demographic information such as country of birth and ethnicity.

\subsection{Health and Retirement Study}

The Health and Retirement Study (HRS) is a cohort of retired American individuals\citep{juster_overview_1995}. We used genotype data from 12,454 individuals from the Health and Retirement Study (HRS), genotyped on the Illumina Human Omni 2.5M platform. The database contains basic demographic data such as age, US Census Bureau region of birth, and race.

\subsection{UK biobank}

The UK biobank (UKB) is a cohort of individuals living in the United Kingdom who were recruited by inviting those registered with the National Health Service (NHS)\citep{sudlow_uk_2015}. It contains the genotypes from $488,377$ participants as well as detailed health data, phenotypic measures, geographic coordinates, and sociodemographic information such as ethnic background.

%\section{Methods}

%\subsection{PCA}

%\subsection{UMAP}

%\subsection{HDBSCAN(\texorpdfstring{$\hat{\epsilon}$}{f})}
% the {f} here doesn't do anything, but we need a text character for compilation

%\subsection{Genomic tools}