The works presented in this thesis establish novel methodologies for dimensionality reduction, clustering, and visualization in population genetics with possible extensions to genomics at large. We explore applications of UMAP, a nonlinear dimensionality-reduction method, and HDBSCAN($\hat{\epsilon}$), a density-based clustering algorithm. The two methods form the basis of our topological data analysis. The approach is tractable, easy-to-implement, and fits in the paradigm of exploratory-confirmatory analysis of biobank data, particularly for large and complex cohorts.

In \hyperref[chap:chapter2]{Chapter~2} we apply UMAP to population genetic data for the first time. We establish that it efficiently reveals fine-scale population structure in biobanks and use it in data visualization in 2- and 3-dimensions. These visualizations correlate with a number of phenotypic, geographic, and socio-demographic measures, and often appear as clusters. As UMAP became popular in the field, in \hyperref[chap:chapter3]{Chapter~3} we review its use in population genetics and provide insights on its optimal use and potential downstream applications.

Finally, in \hyperref[chap:chapter4]{Chapter~4} we address algorithmic clustering of UMAP results. We apply HDBSCAN($\hat{\epsilon}$) to UMAP data in $3$ to $5$ dimensions and use it to stratify biobank data. We discover clusters of structure in genetic data and leverage them to study demographic histories of populations in biobanks, polygenic score transferability, phenotype distributions, identify potentially influential alleles, and its use in quality control.