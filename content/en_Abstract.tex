Advances in DNA sequencing technology have led to the rise of the biobank. Now containing the genetic data of millions of individuals, biobanks are rich repositories, used regularly and fuelling scientific discovery. The human genome spans approximately three billion base pairs, making the study of large-scale genetic data one of very high dimensions. The sheer scale of the problem is challenging, as is the complexity of human genetic diversity. With new methods in topological data analysis, we are able to reduce the dimensionality of these massive data sets while preserving significant amounts of information using a methodology that is tractable on the biobank-scale.

In this research, we apply uniform manifold approximation and projection (UMAP), a form of non-linear dimensionality reduction, and HDBSCAN, a density-based clustering algorithm, to several biobanks of human genetic data. We use UMAP and HDBSCAN to study population structure, the phenomenon in which genetic variation is non-random and correlated with factors like geography, demographic history, migration, and social structure. We develop a methodology in which we can visualize genetic data, identify structure in genetic variation ranging from a handful to hundreds of thousands of individuals. We uncover subtle relationships between genetics, history, geography, and phenotype distributions.