Avec les progrès de la génomique, les biobanques sont en plein essor. Contenant les données génétiques de millions d'individus, les biobanques sont utilisées régulièrement et permettent de nombreuses découvertes scientifiques. Le génome humain s'étend sur environ trois milliards de paires de bases, ce qui fait de l'étude des données génétiques à grande échelle une tâche dans un espace de grande dimension. La taille des données et la richesse de la diversité génétique humaine compliquent l'analyse des données. L'objectif de cette thèse est de développer des méthodes basées sur la topologie pour transformer ces données en une version en plus basse dimension qui préserve le plus d'information possible. 

Dans le cadre de cette recherche, nous appliquons «uniform manifold approximation and projection» (UMAP), une forme de la réduction de la dimension non linéaire, et HDBSCAN, un algorithme de regroupement basé sur la densité des données, à plusieurs biobanques de données génétiques humaines. Nous utilisons UMAP et HDBSCAN pour étudier la structure de la population, le phénomène dans lequel la variation génétique n'est pas aléatoire et est corrélée à des facteurs tels que la géographie, l'histoire démographique, la migration et la structure sociale. Nous développons une méthodologie qui nous permet de visualiser les données génétiques, d'identifier la structure de la variation génétique allant d'une poignée à des centaines de milliers d'individus. Nous découvrons des relations subtiles entre la génétique, l'histoire, la géographie et la distribution des phénotypes.