Les progrès de la génomique ont conduit à l'essor des biobanques. Contenant désormais les données génétiques de millions d'individus, les biobanques sont de riches dépôts, utilisés régulièrement et alimentant la découverte scientifique. Le génome humain s'étend sur environ trois milliards de paires de bases, ce qui fait de l'étude des données génétiques à grande échelle une tâche de très grande envergure. L'échelle même du problème est un défi, tout comme la complexité de la diversité génétique humaine. Grâce à de nouvelles méthodes d'analyse topologique des données, nous sommes en mesure de réduire la dimensionnalité de ces ensembles massifs de données tout en préservant d'importantes quantités d'informations à l'aide d'une méthodologie traçable à l'échelle de la biobanque.

Dans le cadre de cette recherche, nous appliquons uniform manifold approximation and projection (UMAP), une forme de la réduction de la dimensionnalité non linéaire, et HDBSCAN, un algorithme de regroupement basé sur la densité des données, à plusieurs biobanques de données génétiques humaines. Nous utilisons UMAP et HDBSCAN pour étudier la structure de la population, le phénomène dans lequel la variation génétique n'est pas aléatoire et est corrélée à des facteurs tels que la géographie, l'histoire démographique, la migration et la structure sociale. Nous développons une méthodologie qui nous permet de visualiser les données génétiques, d'identifier la structure de la variation génétique allant d'une poignée à des centaines de milliers d'individus. Nous découvrons des relations subtiles entre la génétique, l'histoire, la géographie et la distribution des phénotypes.